\chapter{Domain Adaptation - Related Work}

Domain adaptation is a well studied problem in machine learning and natural language processing. Usually, there is a plenty of labeled data available for one domain (also known as the source domain) but nearly not enough or in some cases, none available from a different domain (also known as target). The challenge in that case, is to apply well performing systems from source domain and adapt it the target domain. In most problems, this results in a drop in performance, sometimes severe. The same holds true for POS tagging and dependency parsing. These systems perform well when trained and tested on datasets that are predominantly in the same text domain. However, there is a considerable decrease in accuracy if the domains under consideration, are markedly different. It is a well-studied problem for POS tagging and dependency parsing (more so for dependency parsing) but the improvement proposed by these systems have been negligible at best. This is mainly due to unavailability of annotated data from target domain. \cite{daume:07} notes that this problem is ``frustratingly easy'' when some annotated data  is available from the target domain  and ``frustratingly hard'' if no such target data is available~\citep{dredze:blitzer:ea:07}. 

Previous work in this area have largely focused on the classic domain adaptation problem. The closest comparison to my approach is the one by \cite{plank2011effective}. However, the problem they address is creating a specialized training set for every document they need to parse. They pick sentences from the training set which are most similar to test set. Topic distribution is thus used as features for similarity metrics. My approach is a more general, because I create more general domain training ``experts". My approach also draws parallel with the work on ``multiple source parse adaptation" by \cite{mcclosky2010automatic}. In this approach the parser is trained on multiple domains and learned the statistics as well as  domain differences which affect the parser accuracy. This is similar to my approach as I create experts based on topics , and each expert learns the specifics of the particular topic with which it is associated.